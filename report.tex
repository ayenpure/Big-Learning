\documentclass[12pt]{extarticle}
\usepackage[utf8]{inputenc}
\usepackage{cite}

\title{Big Learning Seminar Report}
\author{Abhishek Yenpure}
\date{Spring 2019}

\begin{document}

\maketitle

\section{Introduction}
%
The intent of this report is to discuss the learnings of the seminar.
%
The seminar focused on the area of intersection of two fields, Big Data and Deep Learning.
%
In this section I introduce the readers to the broad themes of these two fields.
%
This introduction helps build a foundation that helps readers to better understand the
sections in which describes works discussed in the seminar. 

Big Data and Deep Learning are among the most popular research fields.
%

%
\subsection{Big Data}
"Big Data" usually referes to ways to analyze and extract information form very
data that is too large.
%
Big Data processing has become increasingly important due to the amounts of data
that is being generated on a daily basis.
%
Processing such large data is a challange because traditonal approaches often fail
to do so efficiently.
%
Big data is characterized by three Vs, namely Volume, Velocity, and Variety.
%
Each of these three Vs represents different challanges involved in the processing
of big data.
%
\begin{itemize}
\item \textbf{Volume:}
%
This aspect of big data alludes to the magnitudes of data that needs to be stored
and processed.
%
The fact that 90\% of the world's data was generated in the past two years gives
us an idea of why dealing with this aspect of big data is necessary.
%
To be able to make the most use of available data we need to process huge volumes
of data, most times with a time constraint.
%
This further aggravated the challange.

\item \textbf{Velocity:}

\item \textbf{Variety:}

\end{itemize}
%
\subsection{Deep Learning}
 

\section{GPUs}
Cite this~\cite{chen2014big}

\section{Traffic Modeling}

\section{Something random}

\bibliographystyle{plain}
\bibliography{report}

\end{document}
